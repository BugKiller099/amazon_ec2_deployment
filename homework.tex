\documentclass[a4paper,12pt]{article}
\usepackage[utf8]{inputenc}
\usepackage{listings}
\usepackage{xcolor}
\usepackage{hyperref}

\lstset{
    basicstyle=\ttfamily\small,
    keywordstyle=\color{blue},
    stringstyle=\color{red},
    commentstyle=\color{green!50!black},
    backgroundcolor=\color{gray!10},
    numbers=left,
    numberstyle=\tiny\color{gray},
    breaklines=true,
    frame=single,
    captionpos=b,
}

\title{Express Application Setup and npm Versioning Guide}
\author{}
\date{}

\begin{document}

\maketitle

\section*{1. Create a Repository}
\begin{itemize}
    \item Create a directory for your project:
    \begin{lstlisting}[language=bash]
    mkdir my-express-app
    cd my-express-app
    \end{lstlisting}
    \item Initialize a Git repository:
    \begin{lstlisting}[language=bash]
    git init
    \end{lstlisting}
\end{itemize}

\section*{2. Initialize the Repository (npm init)}
\begin{itemize}
    \item Run the following command to initialize the project with \texttt{package.json}:
    \begin{lstlisting}[language=bash]
    npm init -y
    \end{lstlisting}
    \item This creates a default \texttt{package.json} file.
\end{itemize}

\section*{3. Install Required Files}
\begin{itemize}
    \item The following files will be created automatically:
    \begin{itemize}
        \item \texttt{node\_modules/}: Contains installed packages.
        \item \texttt{package.json}: Manages project metadata and dependencies.
        \item \texttt{package-lock.json}: Locks dependencies to specific versions.
    \end{itemize}
\end{itemize}

\section*{4. Install Express}
\begin{itemize}
    \item Install the Express library:
    \begin{lstlisting}[language=bash]
    npm install express
    \end{lstlisting}
\end{itemize}

\section*{5. Create a Server}
\begin{itemize}
    \item Create a file named \texttt{server.js} and add the following code:
    \begin{lstlisting}[language=javascript]
    const express = require('express');
    const app = express();

    app.get('/test', (req, res) => {
        res.send('This is the /test route');
    });

    app.listen(9999, () => {
        console.log('Server is running on port 9999');
    });
    \end{lstlisting}
    \item Run the server:
    \begin{lstlisting}[language=bash]
    node server.js
    \end{lstlisting}
\end{itemize}

\section*{6. Install Nodemon and Update Scripts}
\begin{itemize}
    \item Install \texttt{nodemon} as a development dependency:
    \begin{lstlisting}[language=bash]
    npm install nodemon --save-dev
    \end{lstlisting}
    \item Update the \texttt{scripts} section of your \texttt{package.json} file:
    \begin{lstlisting}[language=json]
    "scripts": {
        "start": "node server.js",
        "dev": "nodemon server.js"
    }
    \end{lstlisting}
    \item Start the server in development mode using:
    \begin{lstlisting}[language=bash]
    npm run dev
    \end{lstlisting}
\end{itemize}

\section*{7. Difference Between Caret (\texttt{\^}) and Tilde (\texttt{\~}) in Versioning}
\begin{itemize}
    \item \textbf{Caret (\texttt{\^})}:
    \begin{itemize}
        \item Allows updates to the most recent \textit{minor} and \textit{patch} versions within the same \textit{major} version.
        \item Example: \texttt{\^4.5.6} allows \texttt{4.5.6}, \texttt{4.6.0}, \texttt{4.7.0}, but not \texttt{5.0.0}.
    \end{itemize}
    \item \textbf{Tilde (\texttt{\~})}:
    \begin{itemize}
        \item Allows updates to the most recent \textit{patch} version within the same \textit{minor} version.
        \item Example: \texttt{\~4.5.6} allows \texttt{4.5.6}, \texttt{4.5.7}, \texttt{4.5.8}, but not \texttt{4.6.0}.
    \end{itemize}
\end{itemize}

\section*{8. Use of \texttt{-g} When Installing npm Packages}
\begin{itemize}
    \item The \texttt{-g} flag stands for \textbf{global installation}.
    \item When a package is installed globally:
    \begin{itemize}
        \item It can be used from any directory on your system.
        \item Useful for command-line tools like \texttt{nodemon} or \texttt{npm}.
    \end{itemize}
    \item Example:
    \begin{lstlisting}[language=bash]
    npm install -g nodemon
    \end{lstlisting}
    \item After installation, you can use \texttt{nodemon} directly in your terminal:
    \begin{lstlisting}[language=bash]
    nodemon server.js
    \end{lstlisting}
\end{itemize}

\section*{Summary}
- You now have an Express app that:
    \begin{itemize}
        \item Serves a \texttt{/test} route.
        \item Uses \texttt{nodemon} for auto-restarting the server in development.
        \item Properly manages versioning with caret (\texttt{\^}) and tilde (\texttt{\~}).
    \end{itemize}

\end{document}
